% Re:VIEWの場合サブタイトルは使わない前提
% ---------------------------------------------
simplesquarebox

\begin{simplesquarebox}{CAPTION}

TEST, TEST, TEST

\end{simplesquarebox}

% FIXME:キャプションなしの表現がおかしい
\begin{simplesquarebox}{}

TEST, TEST, TEST

\end{simplesquarebox}


% オプションとしてthicknessを持つ
\begin{simplesquarebox}{CAPTION}[1.5]

TEST, TEST, TEST

\end{simplesquarebox}

% review側からは\begin{note}[CAPTION] になる。

% オプションでやりづらい… CAPTION、固有オプション、tcolorboxオプション で指定できるものとする。

\begin{reviewnote}

TEST, TEST, TEST

\end{reviewnote}

\begin{reviewnote}[CAPTION]

TEST, TEST, TEST

\end{reviewnote}

\begin{reviewcaution}[CAPTION]

TEST, TEST, TEST

\end{reviewcaution}

\begin{reviewcaution}

TEST, TEST, TEST

\end{reviewcaution}

\begin{reviewnote}[CAPTION]

TEST, TEST, TEST

\end{reviewnote}

\begin{rv@ascolorbox1@caption}{キャプション}

TEST, TEST, TEST

\end{rv@ascolorbox1@caption}

\begin{rv@ascolorbox1@nocaption}

TEST, TEST, TEST

\end{rv@ascolorbox1@nocaption}

\begin{rv@ascolorbox2@caption}{キャプション}[rv line color=red]

TEST, TEST, TEST

\end{rv@ascolorbox2@caption}

\begin{rv@ascolorbox2@nocaption}

TEST, TEST, TEST

\end{rv@ascolorbox2@nocaption}

\begin{rv@ascolorbox3@caption}{キャプション}[rv line color=red]

TEST, TEST, TEST

\end{rv@ascolorbox3@caption}

\begin{rv@ascolorbox3@nocaption}[rv dummy caption=ほげほげほげほげ, rv line color=red]

TEST, TEST, TEST

\end{rv@ascolorbox3@nocaption}



% \begin{rvsimplesquarebox}{CAPTION}{thickness=1.5}{colframe=red!75!black}
% 
% TEST, TEST, TEST
% 
% \end{rvsimplesquarebox}
% 
% \begin{rvsimplesquarebox}{}{thickness=1.5}{}
% 
% TEST, TEST, TEST
% 
% \end{rvsimplesquarebox}
% 
% \begin{rvsimplesquarebox}{}{}{}
% 
% TEST, TEST, TEST
% 
% \end{rvsimplesquarebox}

% ---------------------------------------------
practicebox
% FIXME:1行の内容のときにおかしい
\begin{practicebox}{CAPTION}

TEST, TEST, TEST

\end{practicebox}

\begin{practicebox}{}

TEST, TEST, TEST

\end{practicebox}

% ---------------------------------------------
ascolorbox1

\begin{ascolorbox1}{CAPTION}

TEST, TEST, TEST

\end{ascolorbox1}

% FIXME:キャプションなしのときにおかしい
\begin{ascolorbox1}{}

TEST, TEST, TEST

\end{ascolorbox1}

% ---------------------------------------------
ascolorbox2

\begin{ascolorbox2}{CAPTION}

TEST, TEST, TEST

\end{ascolorbox2}

\begin{ascolorbox2}{}

TEST, TEST, TEST

\end{ascolorbox2}

% ---------------------------------------------
ascolorbox3

\begin{ascolorbox3}{CAPTION}

TEST, TEST, TEST

\end{ascolorbox3}

% FIXME:キャプションなしのときにちょっとおかしい
\begin{ascolorbox3}{}

TEST, TEST, TEST

\end{ascolorbox3}

% オプションとして枠色を持つ
\begin{ascolorbox3}{CAPTION}[orange]

TEST, TEST, TEST

\end{ascolorbox3}

% ---------------------------------------------
ascolorbox4

\begin{ascolorbox4}{CAPTION}

TEST, TEST, TEST

\end{ascolorbox4}

% FIXME:キャプションなしのときにおかしい
\begin{ascolorbox4}{}

TEST, TEST, TEST

\end{ascolorbox4}

% オプションとして幅・半径を持つ
\begin{ascolorbox4}{CAPTION}[2]

TEST, TEST, TEST

\end{ascolorbox4}

% ---------------------------------------------
ascolorbox5

\begin{ascolorbox5}{CAPTION}

TEST, TEST, TEST

\end{ascolorbox5}

% FIXME:キャプションなしのときにエラー
%\begin{ascolorbox5}{ }

%TEST, TEST, TEST

%\end{ascolorbox5}

% 色を変える
\begin{ascolorbox5}{CAPTION}[orange]

TEST, TEST, TEST

\end{ascolorbox5}

% ---------------------------------------------
ascolorbox8

\begin{ascolorbox8}{CAPTION}

TEST, TEST, TEST

\end{ascolorbox8}

\begin{ascolorbox8}{}

TEST, TEST, TEST

\end{ascolorbox8}

% ---------------------------------------------
ascolorbox9

\begin{ascolorbox9}{CAPTION}

TEST, TEST, TEST

\end{ascolorbox9}

% FIXME:キャプションなしのときにエラー
%\begin{ascolorbox9}{}

%TEST, TEST, TEST

%\end{ascolorbox9}

% 丸模様の繰り返し回数
\begin{ascolorbox9}{CAPTION}[5]

TEST, TEST, TEST

\end{ascolorbox9}

% ---------------------------------------------
ascolorbox10

\begin{ascolorbox10}{CAPTION}

TEST, TEST, TEST

\end{ascolorbox10}

% FIXME:キャプションなしの表現がおかしい
\begin{ascolorbox10}{}

TEST, TEST, TEST

\end{ascolorbox10}

% オプションとしてthicknessを持つ
\begin{ascolorbox10}{CAPTION}[1.5]

TEST, TEST, TEST

\end{ascolorbox10}

% ---------------------------------------------
ascolorbox11

\begin{ascolorbox11}{CAPTION}

TEST, TEST, TEST

\end{ascolorbox11}

% FIXME:キャプションなしの表現がおかしい
\begin{ascolorbox11}{}

TEST, TEST, TEST

\end{ascolorbox11}

% オプションとして半径を持つ
\begin{ascolorbox11}{CAPTION}[2]

TEST, TEST, TEST

\end{ascolorbox11}

% ---------------------------------------------
ascolorbox12

\begin{ascolorbox12}{CAPTION}

TEST, TEST, TEST

\end{ascolorbox12}

% FIXME:キャプションなしの表現がちょっとおかしい
\begin{ascolorbox12}{}

TEST, TEST, TEST

\end{ascolorbox12}

% ---------------------------------------------
ascolorbox13

\begin{ascolorbox13}{CAPTION}

TEST, TEST, TEST

\end{ascolorbox13}

\begin{ascolorbox13}{}

TEST, TEST, TEST

\end{ascolorbox13}

% ---------------------------------------------
ascolorbox14

\begin{ascolorbox14}{CAPTION}

TEST, TEST, TEST

\end{ascolorbox14}

% FIXME:キャプションなしの表現がちょっとおかしい
\begin{ascolorbox14}{}

TEST, TEST, TEST

\end{ascolorbox14}

% ---------------------------------------------
ascolorbox15

\begin{ascolorbox15}{CAPTION}

TEST, TEST, TEST

\end{ascolorbox15}

% FIXME:キャプションなしの表現がちょっとおかしい
\begin{ascolorbox15}{}

TEST, TEST, TEST

\end{ascolorbox15}

% ---------------------------------------------
ascolorbox16

\begin{ascolorbox16}{CAPTION}

TEST, TEST, TEST

\end{ascolorbox16}

% FIXME:キャプションなしの表現がおかしい
\begin{ascolorbox16}{}

TEST, TEST, TEST

\end{ascolorbox16}

% ---------------------------------------------
ascolorbox17

\begin{ascolorbox17}{CAPTION}

TEST, TEST, TEST

\end{ascolorbox17}

% FIXME:キャプションなしの表現がおかしい
\begin{ascolorbox17}{}

TEST, TEST, TEST

\end{ascolorbox17}

% オプションとして枠色変更を持つ
\begin{ascolorbox17}{CAPTION}[orange]

TEST, TEST, TEST

\end{ascolorbox17}

% ---------------------------------------------
ascolorbox18

\begin{ascolorbox18}{CAPTION}

TEST, TEST, TEST

\end{ascolorbox18}

\begin{ascolorbox18}{}

TEST, TEST, TEST

\end{ascolorbox18}

\begin{ascolorbox19}{CAPTION}

TEST, TEST, TEST

% ---------------------------------------------
ascolorbox19

\end{ascolorbox19}

% FIXME:キャプションなしの表現がおかしい
\begin{ascolorbox19}{}

TEST, TEST, TEST

\end{ascolorbox19}

% オプションとして間隔を持つ
\begin{ascolorbox19}{CAPTION}[5]

TEST, TEST, TEST

\end{ascolorbox19}

% 普通にシンプルなもの、二重囲み程度のものも追加で用意する
% ascboxを見出しに付ける方法もほしくなりそう
